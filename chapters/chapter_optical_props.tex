\chapter{Properties of Single-Wall Carbon Nanotubes}

\section{Introduction}
{\color{red}UNFINISHED} Carbon nanotubes are 1-D structures meaning that electrons are confined in a single dimension. They exist in terms of many chiralities, some are semiconducting and others are metallic \cite{soavi2016ultrafast}. Interesting because they let us explore the physics of 1-D structures. Due to the higher degree of confinement with respect to convention 3-D structures, different phenomena may occur.



\section{Carbon Nanotube Chiralities}

Carbon nanotubes are basically rolled up graphene sheets. Countless ways of rolling up a graphene sheet into cylindrical structures. Hence, nanotubes described using so-called chiral vectors. Species of carbon nanotubes are denoted using a set of indices ($m$,$n$). These integers $m$ and $n$ define the chiral vector $\vec{C_h }$ expressed as 
\begin{equation}
	\vec{C_h} = n \bm{\vec{a_1}} + m \bm{\vec{a_2}}
\end{equation}

where $\vec{a_1}$ and $\vec{a_2}$ represent the basis vectors of the 2D graphene sheet as shown in Figure \ref{fig:chiral_vectors} \cite{nanot2012optoelectronic}. In general, (n,n) carbon nanotubes constitute the set of all metallic nanotubes \cite{nanot2012optoelectronic}. All (n,m) nanotubes where $n-m = 3j$ ($j > 0$) comprise a set of small-gap ($\sim1 - 100$ meV) semiconductors \cite{nanot2012optoelectronic}. The remaining nanotubes outside of these two categories include medium-gap semiconductors ($\sim0.5 - 1$ eV) \cite{nanot2012optoelectronic}.

\begin{figure}[H]
	\centering
	\includegraphics[scale=0.7]{images/chapter_optical_props/chiral_vectors.png}
	\caption{{\color{red}UNFINISHED CAPTION} Reproduced from \cite{odom2000structure}.}
	\label{fig:chiral_vectors}
\end{figure}


\section{Optical Selection Rules}

Selection rules dictate the optical transitions that can occur under certain conditions. Light polarized parallel to carbon nanotube excites one type of transition. Whereas, light polarized perpendicular to carbon nanotube excites another set of transitions.


\begin{figure}[H]
	\centering
	\begin{subfigure}{\textwidth}
		\centering
		\includegraphics[scale=0.7]{images/chapter_optical_props/selection_rules_1.png}
		\caption{Selection Rules for Parallel case.}
	\end{subfigure}
	\begin{subfigure}{\textwidth}
		\centering
		\includegraphics[scale=0.7]{images/chapter_optical_props/selection_rules_2.png}
		\caption{Selection rules for perpendicular case}
	\end{subfigure}
	\caption{Selection rules. Reproduced from Ref \cite{ando2005theory}.}
	\label{fig:selection_rules}
\end{figure}

\section{Excitons in Carbon Nanotubes}
All optical excitations in carbon nanotubes lead to direct creation of excitons \cite{wang2005optical}.

Due to strong quantum confinement, binding energy of Excitons in 1-D structures expected to be infinite \cite{ando2005theory}. This explains why nanotubes have such high exciton binding compared to 2-D and 3-D materials. 
 
\begin{figure}[h]
	\centering
	\includegraphics[scale=0.7]{example-image-a}
	\caption{Figure of GaAs absorbance at low-T and high-T to show weak binding energy of excitons. Next to this is a figure of  (6,5) absorbance at room temperature}
	\label{fig:gaas_vs_cnt_absorbance}
\end{figure}
