\chapter{Optical Properties of Single-Wall Carbon Nanotubes}

\section{Introduction}
Carbon nanotubes are:
\begin{itemize}
	\item 1-D structures meaning that electrons are confined in a single dimension
	\item exist in terms of many chiralities, some are semiconducting and others are metallic
	\item interesting because they let us explore the physics of 1-D structures. 
	\item Due to the higher degree of confinement with respect to convention 3-D structures, different phenomena may occur.
\end{itemize}

\section{Optical Selection Rules}
\begin{itemize}
	\item Selection rules dictate which transitions can occur in the presence of certain conditions
	\item light polarized parallel to carbon nanotube excites one type of transition
	\item light polarized perpendicular to carbon nanotube excites another set of transitions
\end{itemize}

\begin{figure}[h]
	\centering
	\includegraphics[scale=0.7]{example-image-a}
	\caption{Figure of carbon nanotube bandstructure. Arrows drawn in figure to show allowed transitions.}
	\label{fig:beam_diamter_measurement}
\end{figure}

\section{Excitons in Carbon Nanotubes}
\begin{itemize}
	\item all optical excitations in carbon nanotubes lead to direct creation of excitons
\end{itemize}

\begin{figure}[h]
	\centering
	\includegraphics[scale=0.7]{example-image-a}
	\caption{Figure of GaAs absorbance at low-T and high-T to show weak binding energy of excitons. Next to this is a figure of  (6,5) absorbance at room temperature}
	\label{fig:gaas_vs_cnt_absorbance}
\end{figure}
