\chapter{Conclusions}

In this thesis, we used optical pump-probe spectroscopy to study coherent and incoherent phenomena in (6,5) SWCNTs. We observed a clear signature of optical Stark effect in (6,5) SWCNTs. When (6,5) SWCNTs were optically excited at a photon energy below the band gap, a coherent blue shift of the lowest exciton resonance occurred. However, we also noted that real carriers were created via multi-photon processes. The dynamics of dephasing for these carriers were best explained by invoking the concept of quantum memory, rather than a phenomenological approach. Here we noted that these fast dynamics make SWCNTs suitable for ultrafast all-optical switching applications based on the optical Stark effect.

Furthermore, we observed evidence of exciton quenching for (6,5) SWCNTs suspended using sodium deoxycholate (DOC) as a surfactant but not in (6,5) SWCNTs that were suspended using the polymer PFO-BPy. In both experiments, the (6,5) SWCNTs were resonantly photoexcited at the second-highest exciton resonance. We posited that the difference in the dielectric environment of these samples may have had an effect on the upper limit of excitons that can be created. This result leaves open questions regarding the extent at which different dielectric environments affect exciton quenching in SWCNTs. 
