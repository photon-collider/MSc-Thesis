\chapter{Conclusions}

In this thesis, we used optical pump-probe spectroscopy to study coherent and incoherent phenomena in (6,5) SWCNTs. We observed a clear signature of optical Stark effect in (6,5) SWCNTs. In this scenario, the dynamics of dephasing were best explained by invoking the concept of quantum memory, rather than a phenomenological approach, which plays a role in the fast relaxation of real carriers created via multi-photon processes. Here we noted that these dephasing dynamics make SWCNTs suitable for ultrafast all-optical switching applications based on this phenomenon.

Furthermore, we observed evidence of exciton quenching for (6,5) SWCNTs suspended using sodium deoxycholate (DOC) as a surfactant but not in (6,5) SWCNTs that were suspende using the polymer PFO-BPy. We posited that the difference in the dielectric environment of these samples may have an effect on the upper limit of excitons that can be created. This leaves open questions regarding the extent at which different dielectric environments affect exciton quenching in SWCNTs. 
