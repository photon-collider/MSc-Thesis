\thispagestyle{empty}
\begin{abstract}
Single-wall carbon nanotubes (SWCNTs) represent an exemplary model system for studying one-dimensional (1-D) physics in condensed matter research. Conceptually, they can be presented as a rolled-up sheet of graphene, and depending on their crystal structure they either exhibit properties of semiconductors or metals. These materials also feature strong Coulomb interactions that have profound effects, giving rise to optical properties which differ from those of 2-D and 3-D materials. Chief amongst these properties is the tendency for optical excitations to only generate bound electron-hole pairs in SWCNTs, known as excitons, rather than free electron-hole pairs. In this thesis, we used ultrafast optical pump-probe spectroscopy to investigate both coherent and noncoherent phenomena in (6,5) SWCNTs. The samples used for this investigation included two different ensembles of individually-suspended (6,5)-enriched SWCNTs in solution. In one ensemble, the SWCNTs were suspended in an aqueous solution containing surfactants, whereas the SWCNTs in the other ensemble were suspended using an aromatic polymer in a toluene solution. In this study, we observed a coherent blueshift of the lowest-lying exciton resonance when the SWCNTs were photoexcited below the band gap, a signature of the optical Stark effect. When photoexciting the samples at the second-highest exciton resonance, we observed quenching of the lowest exciton resonance for the SWCNTs in the aqueous suspension but not for those that were suspended in toluene. This observation alone raises further questions regarding how the dielectric environment surrounding
 SWCNTs affects the carrier dynamics that they exhibit.
\end{abstract}
